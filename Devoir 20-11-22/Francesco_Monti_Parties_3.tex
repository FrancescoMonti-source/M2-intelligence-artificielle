% Options for packages loaded elsewhere
\PassOptionsToPackage{unicode}{hyperref}
\PassOptionsToPackage{hyphens}{url}
%
\documentclass[
]{article}
\usepackage{amsmath,amssymb}
\usepackage{lmodern}
\usepackage{iftex}
\ifPDFTeX
  \usepackage[T1]{fontenc}
  \usepackage[utf8]{inputenc}
  \usepackage{textcomp} % provide euro and other symbols
\else % if luatex or xetex
  \usepackage{unicode-math}
  \defaultfontfeatures{Scale=MatchLowercase}
  \defaultfontfeatures[\rmfamily]{Ligatures=TeX,Scale=1}
\fi
% Use upquote if available, for straight quotes in verbatim environments
\IfFileExists{upquote.sty}{\usepackage{upquote}}{}
\IfFileExists{microtype.sty}{% use microtype if available
  \usepackage[]{microtype}
  \UseMicrotypeSet[protrusion]{basicmath} % disable protrusion for tt fonts
}{}
\makeatletter
\@ifundefined{KOMAClassName}{% if non-KOMA class
  \IfFileExists{parskip.sty}{%
    \usepackage{parskip}
  }{% else
    \setlength{\parindent}{0pt}
    \setlength{\parskip}{6pt plus 2pt minus 1pt}}
}{% if KOMA class
  \KOMAoptions{parskip=half}}
\makeatother
\usepackage{xcolor}
\usepackage[margin=1in]{geometry}
\usepackage{color}
\usepackage{fancyvrb}
\newcommand{\VerbBar}{|}
\newcommand{\VERB}{\Verb[commandchars=\\\{\}]}
\DefineVerbatimEnvironment{Highlighting}{Verbatim}{commandchars=\\\{\}}
% Add ',fontsize=\small' for more characters per line
\usepackage{framed}
\definecolor{shadecolor}{RGB}{248,248,248}
\newenvironment{Shaded}{\begin{snugshade}}{\end{snugshade}}
\newcommand{\AlertTok}[1]{\textcolor[rgb]{0.94,0.16,0.16}{#1}}
\newcommand{\AnnotationTok}[1]{\textcolor[rgb]{0.56,0.35,0.01}{\textbf{\textit{#1}}}}
\newcommand{\AttributeTok}[1]{\textcolor[rgb]{0.77,0.63,0.00}{#1}}
\newcommand{\BaseNTok}[1]{\textcolor[rgb]{0.00,0.00,0.81}{#1}}
\newcommand{\BuiltInTok}[1]{#1}
\newcommand{\CharTok}[1]{\textcolor[rgb]{0.31,0.60,0.02}{#1}}
\newcommand{\CommentTok}[1]{\textcolor[rgb]{0.56,0.35,0.01}{\textit{#1}}}
\newcommand{\CommentVarTok}[1]{\textcolor[rgb]{0.56,0.35,0.01}{\textbf{\textit{#1}}}}
\newcommand{\ConstantTok}[1]{\textcolor[rgb]{0.00,0.00,0.00}{#1}}
\newcommand{\ControlFlowTok}[1]{\textcolor[rgb]{0.13,0.29,0.53}{\textbf{#1}}}
\newcommand{\DataTypeTok}[1]{\textcolor[rgb]{0.13,0.29,0.53}{#1}}
\newcommand{\DecValTok}[1]{\textcolor[rgb]{0.00,0.00,0.81}{#1}}
\newcommand{\DocumentationTok}[1]{\textcolor[rgb]{0.56,0.35,0.01}{\textbf{\textit{#1}}}}
\newcommand{\ErrorTok}[1]{\textcolor[rgb]{0.64,0.00,0.00}{\textbf{#1}}}
\newcommand{\ExtensionTok}[1]{#1}
\newcommand{\FloatTok}[1]{\textcolor[rgb]{0.00,0.00,0.81}{#1}}
\newcommand{\FunctionTok}[1]{\textcolor[rgb]{0.00,0.00,0.00}{#1}}
\newcommand{\ImportTok}[1]{#1}
\newcommand{\InformationTok}[1]{\textcolor[rgb]{0.56,0.35,0.01}{\textbf{\textit{#1}}}}
\newcommand{\KeywordTok}[1]{\textcolor[rgb]{0.13,0.29,0.53}{\textbf{#1}}}
\newcommand{\NormalTok}[1]{#1}
\newcommand{\OperatorTok}[1]{\textcolor[rgb]{0.81,0.36,0.00}{\textbf{#1}}}
\newcommand{\OtherTok}[1]{\textcolor[rgb]{0.56,0.35,0.01}{#1}}
\newcommand{\PreprocessorTok}[1]{\textcolor[rgb]{0.56,0.35,0.01}{\textit{#1}}}
\newcommand{\RegionMarkerTok}[1]{#1}
\newcommand{\SpecialCharTok}[1]{\textcolor[rgb]{0.00,0.00,0.00}{#1}}
\newcommand{\SpecialStringTok}[1]{\textcolor[rgb]{0.31,0.60,0.02}{#1}}
\newcommand{\StringTok}[1]{\textcolor[rgb]{0.31,0.60,0.02}{#1}}
\newcommand{\VariableTok}[1]{\textcolor[rgb]{0.00,0.00,0.00}{#1}}
\newcommand{\VerbatimStringTok}[1]{\textcolor[rgb]{0.31,0.60,0.02}{#1}}
\newcommand{\WarningTok}[1]{\textcolor[rgb]{0.56,0.35,0.01}{\textbf{\textit{#1}}}}
\usepackage{graphicx}
\makeatletter
\def\maxwidth{\ifdim\Gin@nat@width>\linewidth\linewidth\else\Gin@nat@width\fi}
\def\maxheight{\ifdim\Gin@nat@height>\textheight\textheight\else\Gin@nat@height\fi}
\makeatother
% Scale images if necessary, so that they will not overflow the page
% margins by default, and it is still possible to overwrite the defaults
% using explicit options in \includegraphics[width, height, ...]{}
\setkeys{Gin}{width=\maxwidth,height=\maxheight,keepaspectratio}
% Set default figure placement to htbp
\makeatletter
\def\fps@figure{htbp}
\makeatother
\setlength{\emergencystretch}{3em} % prevent overfull lines
\providecommand{\tightlist}{%
  \setlength{\itemsep}{0pt}\setlength{\parskip}{0pt}}
\setcounter{secnumdepth}{-\maxdimen} % remove section numbering
\ifLuaTeX
  \usepackage{selnolig}  % disable illegal ligatures
\fi
\IfFileExists{bookmark.sty}{\usepackage{bookmark}}{\usepackage{hyperref}}
\IfFileExists{xurl.sty}{\usepackage{xurl}}{} % add URL line breaks if available
\urlstyle{same} % disable monospaced font for URLs
\hypersetup{
  pdftitle={Homework part 3},
  pdfauthor={Francesco MONTI},
  hidelinks,
  pdfcreator={LaTeX via pandoc}}

\title{Homework part 3}
\author{Francesco MONTI}
\date{2022-11-17}

\begin{document}
\maketitle

\begin{Shaded}
\begin{Highlighting}[]
\FunctionTok{library}\NormalTok{(stringr)}
\FunctionTok{library}\NormalTok{(prettyR)}
\FunctionTok{library}\NormalTok{(ggplot2)}
\FunctionTok{library}\NormalTok{(ggthemes)}
\FunctionTok{library}\NormalTok{(colorRamps)}

\NormalTok{knitr}\SpecialCharTok{::}\NormalTok{opts\_chunk}\SpecialCharTok{$}\FunctionTok{set}\NormalTok{(}\AttributeTok{echo =}\NormalTok{ F, }\AttributeTok{warning =}\NormalTok{ F) }\CommentTok{\# setting chunk options globally}
\end{Highlighting}
\end{Shaded}

\hypertarget{import-the-file-bdd_vican.csv}{%
\subsection{1) Import the file
``BDD\_VICAN.csv''}\label{import-the-file-bdd_vican.csv}}

\hypertarget{display-the-first-lines-of-the-dataset.-display-the-lines-1-4-18-103-of-the-dataset}{%
\subsection{2) Display the first lines of the dataset. Display the lines
1; 4; 18; 103 of the
dataset}\label{display-the-first-lines-of-the-dataset.-display-the-lines-1-4-18-103-of-the-dataset}}

\begin{verbatim}
##   fc_caisse ms_codcancer fc_agediag_r0 q5_sd4_r1 q5_sd5 q5_sd10_r2 q5_pcs12_r1
## 1         1            1            51         1      1          3    36.17192
## 2         1            1            50         2      1          2    44.73504
## 3         1            1            49         2      2          2    58.32207
## 4         1            7            50         1      1          3    56.38174
## 5         1           10            26         1      1          3    61.12935
## 6         1           81            35         1      1          3    20.59613
##   q5_mcs12_r1 q5_eortc_fatigue_r1 q5_anxiete q5_depression q5_jobv5.36_r1
## 1    32.31987           66,666667          1             0              3
## 2    33.16873           66,666667          2             0              1
## 3    58.63898           22,222222          0             0              2
## 4    51.82826           11,111111          0             0              2
## 5    44.05385           55,555556          1             0              2
## 6    52.33494                 100          0             0              1
##   ms_csp_enq_3c_r1 q5_med23.1 id q5_pain
## 1                3          5  2       0
## 2                1          5  3       0
## 3                1          5  4       0
## 4                2          5  5       0
## 5                1          5  6       1
## 6                2          2  7       1
\end{verbatim}

\begin{verbatim}
##     fc_caisse ms_codcancer fc_agediag_r0 q5_sd4_r1 q5_sd5 q5_sd10_r2
## 1           1            1            51         1      1          3
## 4           1            7            50         1      1          3
## 18          1            1            50         2      1          3
## 103         1            3            51         1      1          3
##     q5_pcs12_r1 q5_mcs12_r1 q5_eortc_fatigue_r1 q5_anxiete q5_depression
## 1      36.17192    32.31987           66,666667          1             0
## 4      56.38174    51.82826           11,111111          0             0
## 18     40.27254    28.48350           66,666667          2             1
## 103    62.53169    38.25137           22,222222          1             0
##     q5_jobv5.36_r1 ms_csp_enq_3c_r1 q5_med23.1  id q5_pain
## 1                3                3          5   2       0
## 4                2                2          5   5       0
## 18               1                2          3  19       1
## 103              2                2          4 106       0
\end{verbatim}

\hypertarget{how-many-variables-and-observations-are-there}{%
\subsection{3. How many variables and observations are
there?}\label{how-many-variables-and-observations-are-there}}

\begin{verbatim}
## [1] 16
\end{verbatim}

\begin{verbatim}
## [1] 3962
\end{verbatim}

\hypertarget{does-this-file-contain-any-missing-values}{%
\subsection{4. Does this file contain any missing
values?}\label{does-this-file-contain-any-missing-values}}

\begin{verbatim}
##           fc_caisse        ms_codcancer       fc_agediag_r0           q5_sd4_r1 
##                   0                   0                   0                   0 
##              q5_sd5          q5_sd10_r2         q5_pcs12_r1         q5_mcs12_r1 
##                   0                   0                   0                   0 
## q5_eortc_fatigue_r1          q5_anxiete       q5_depression      q5_jobv5.36_r1 
##                   0                   0                   0                   0 
##    ms_csp_enq_3c_r1          q5_med23.1                  id             q5_pain 
##                   0                   0                   0                   0
\end{verbatim}

\begin{verbatim}
## [1] 0
\end{verbatim}

On a first impression, the dataframe looks to be free of any NAs. We'll
see later that this is not true: the variable ``q5\_eortc\_fatigue\_r1''
as been incorrectly identified as ``character'' as missing values have
been tagged as ``!NULL'' rather than leaving the cells empty.

As a sidenote, there is no description of ``q5\_eortc\_fatigue\_r1'' in
the statement of the homework

\hypertarget{what-is-the-nature-of-the-variables-studied}{%
\subsection{5. What is the nature of the variables
studied?}\label{what-is-the-nature-of-the-variables-studied}}

\begin{verbatim}
## 'data.frame':    3962 obs. of  16 variables:
##  $ fc_caisse          : int  1 1 1 1 1 1 1 1 1 1 ...
##  $ ms_codcancer       : int  1 1 1 7 10 81 10 1 3 1 ...
##  $ fc_agediag_r0      : int  51 50 49 50 26 35 37 47 50 43 ...
##  $ q5_sd4_r1          : int  1 2 2 1 1 1 1 1 1 2 ...
##  $ q5_sd5             : int  1 1 2 1 1 1 1 2 1 2 ...
##  $ q5_sd10_r2         : int  3 2 2 3 3 3 3 3 2 2 ...
##  $ q5_pcs12_r1        : num  36.2 44.7 58.3 56.4 61.1 ...
##  $ q5_mcs12_r1        : num  32.3 33.2 58.6 51.8 44.1 ...
##  $ q5_eortc_fatigue_r1: chr  "66,666667" "66,666667" "22,222222" "11,111111" ...
##  $ q5_anxiete         : int  1 2 0 0 1 0 0 2 0 0 ...
##  $ q5_depression      : int  0 0 0 0 0 0 0 0 0 0 ...
##  $ q5_jobv5.36_r1     : int  3 1 2 2 2 1 1 2 2 1 ...
##  $ ms_csp_enq_3c_r1   : int  3 1 1 2 1 2 2 2 2 1 ...
##  $ q5_med23.1         : int  5 5 5 5 5 2 5 5 1 4 ...
##  $ id                 : int  2 3 4 5 6 7 8 9 10 11 ...
##  $ q5_pain            : int  0 0 0 0 1 1 0 0 1 0 ...
\end{verbatim}

\hypertarget{some-of-the-variables-are-in-the-wrong-format-for-example-a-qualitative-variable-in-numeric-format.-based-on-the-description-of-each-variable-found-at-the-beginning-of-this-exercise-recode-the-variables-into-the-correct-format}{%
\subsection{6. Some of the variables are in the wrong format, for
example, a qualitative variable in ``numeric'' format. Based on the
description of each variable (found at the beginning of this exercise),
recode the variable(s) into the correct
format}\label{some-of-the-variables-are-in-the-wrong-format-for-example-a-qualitative-variable-in-numeric-format.-based-on-the-description-of-each-variable-found-at-the-beginning-of-this-exercise-recode-the-variables-into-the-correct-format}}

\begin{verbatim}
## [1] 6
\end{verbatim}

\hypertarget{definition-of-clinically-significant-fatigue-score-score-40-on-the-fatigue-scale-included-in-the-survey-the-threshold-at-which-a-fatigue-condition-was-shown-to-be-clinically-significant.-create-a-categorical-variable-based-on-this-definition.}{%
\subsection{7. Definition of clinically significant fatigue score: score
\textgreater= 40 on the fatigue scale included in the survey, the
threshold at which a fatigue condition was shown to be clinically
significant. Create a categorical variable based on this
definition.}\label{definition-of-clinically-significant-fatigue-score-score-40-on-the-fatigue-scale-included-in-the-survey-the-threshold-at-which-a-fatigue-condition-was-shown-to-be-clinically-significant.-create-a-categorical-variable-based-on-this-definition.}}

\hypertarget{group-the-modalities-of-the-variable-sequelae-into-3-modalities.}{%
\subsection{8. Group the modalities of the variable sequelae into 3
modalities.}\label{group-the-modalities-of-the-variable-sequelae-into-3-modalities.}}

This new variable, named ``Q5\_med23.1\_rec'' will be considered in the
following analyses instead of ``Q5\_med23.1''.

\hypertarget{display-the-frequency-table-for-this-new-variable.}{%
\subsection{9. Display the frequency table for this new
variable.}\label{display-the-frequency-table-for-this-new-variable.}}

\begin{verbatim}
##   Important sequelae Moderate sequelae No sequelae
## a "907"              "1652"            "1403"     
## b "22.89 %"          "41.7 %"          "35.41 %"
\end{verbatim}

\hypertarget{concerning-age-what-is-the-average-age-of-our-study-population-then-that-of-breast-cancer.}{%
\subsection{10. Concerning age: What is the average age of our study
population, then that of breast
cancer.}\label{concerning-age-what-is-the-average-age-of-our-study-population-then-that-of-breast-cancer.}}

\begin{verbatim}
## [1] 54.71858
\end{verbatim}

\begin{verbatim}
## [1] 50.34656
\end{verbatim}

\hypertarget{determine-the-95-confidence-intervals-ci-for-each-of-the-calculated-means.}{%
\subsubsection{Determine the 95\% confidence intervals (CI) for each of
the calculated
means.}\label{determine-the-95-confidence-intervals-ci-for-each-of-the-calculated-means.}}

\begin{verbatim}
## [1] 54.33283 55.10433
## attr(,"conf.level")
## [1] 0.95
\end{verbatim}

\begin{verbatim}
## [1] 49.72758 50.96554
## attr(,"conf.level")
## [1] 0.95
\end{verbatim}

\hypertarget{calculate-the-variance-standard-deviation-of-the-sample-then-that-of-breast-cancer.}{%
\subsubsection{Calculate the variance, standard deviation of the sample,
then that of breast
cancer.}\label{calculate-the-variance-standard-deviation-of-the-sample-then-that-of-breast-cancer.}}

\begin{verbatim}
## Description of the age variable for the whole population
\end{verbatim}

\begin{verbatim}
## 
##  Numeric 
##      var    sd
## x 153.38 12.38
\end{verbatim}

\begin{verbatim}
## Description of the age variable for the breast cancer population
\end{verbatim}

\begin{verbatim}
## 
##  Numeric 
##      var    sd
## x 112.86 10.62
\end{verbatim}

\hypertarget{draw-a-graph-that-will-represent-the-distribution-of-age-by-location-of-the-pathology.-choose-the-most-appropriate-graph.-export-the-graph-in-a-pdf-format.}{%
\subsection{11. Draw a graph that will represent the distribution of age
by location of the pathology. Choose the most appropriate graph. Export
the graph in a pdf
format.}\label{draw-a-graph-that-will-represent-the-distribution-of-age-by-location-of-the-pathology.-choose-the-most-appropriate-graph.-export-the-graph-in-a-pdf-format.}}

\includegraphics{Francesco_Monti_Parties_3_files/figure-latex/unnamed-chunk-13-1.pdf}
\includegraphics{Francesco_Monti_Parties_3_files/figure-latex/unnamed-chunk-13-2.pdf}

\begin{verbatim}
## `stat_bin()` using `bins = 30`. Pick better value with `binwidth`.
\end{verbatim}

\includegraphics{Francesco_Monti_Parties_3_files/figure-latex/unnamed-chunk-13-3.pdf}

\begin{verbatim}
## `stat_bin()` using `bins = 30`. Pick better value with `binwidth`.
\end{verbatim}

\hypertarget{determine-the-factors-associated-with-physical-and-then-mental-quality-of-life-including-variables-with-a-p-value-0.2.-which-model-will-you-use-how-will-you-proceed-interpret-the-final-result.}{%
\subsection{12. Determine the factors associated with physical and then
mental quality of life, including variables with a p-value \textless{}
0.2. Which model will you use? How will you proceed? Interpret the final
result.}\label{determine-the-factors-associated-with-physical-and-then-mental-quality-of-life-including-variables-with-a-p-value-0.2.-which-model-will-you-use-how-will-you-proceed-interpret-the-final-result.}}

\hypertarget{export-the-new-database-in-.csv-format.}{%
\subsection{13. Export the new database in ``.csv''
format.}\label{export-the-new-database-in-.csv-format.}}

\hypertarget{section}{%
\subsection{}\label{section}}

\end{document}
